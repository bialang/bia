\chapter{Virtuelle Maschine}
Die virtuelle Maschine bildet eine generische mit einem Stack. Während der Ausführung finden keine Typüberprüfungen ab.

\section{Typen}
Die verschiedenen verfügbaren nativen Datentypen sind folgende:
\begin{itemize}
\item \textit{int}, \textit{uint}, \textit{int8}, \textit{uint8}, \textit{int16}, \textit{uint16}, \textit{int32}, \textit{uint32}, \textit{int64}, \textit{uint64}
\item \textit{float32}, \textit{float64}
\item \textit{string}
\item \textit{regex}
\item \textit{void}
\item Speicheradressen für Daten und Strukturen, die auf dem Heap allokiert wurden
\end{itemize}

\section{Operationen}
\begin{itemize}
\item \textit{push} --- erstellt ein neues Element von einem nativen Datentype auf dem Stapel.
\item \textit{pop} --- löscht eine bestimmte Anzahl an Elementen.
\end{itemize}

\subsection{Push}
Mit der \textit{push}-Operation werden neue Elemente auf den Stapel abgelegt. Diese Operation nimmt zusätzlich eine 

\subsection{Pop}
Mit der \textit{Pop}-Operation wird eine bestimmte Anzahl an Elementen vom Stapel entfernt.


